\documentclass[a4paper, 12pt]{article}

\usepackage[utf8]{inputenc} %define a língua do documento
\usepackage[brazil]{babel} % define a língua do documento
\usepackage[T1]{fontenc} % Para acentuação

% Definição do tamanho da margem de acordo com a ABNT
\usepackage[lmargin=3cm, tmargin=3cm, rmargin=2cm, bmargin=2cm]{geometry}

% Habilita as funções do modo matemático, para utilização de símbolos matemáticos, para usar a função cancel{}
\usepackage{amsmath, amssymb, cancel}

% Usado para fazer as formas geométricas e formar ângulos
\usepackage{tikz, tkz-euclide}


\begin{document}
%%%%%%%%%%%%%%%%%%%%%%%%%%%%%%%%%%%%% Página 1 %%%%%%%%%%%%%%%%%%%%%%%%%%%%%%%%%%%%%%%%
    \section{APROXIMAÇÕES PARA $\pi$}
    \textbf{Definição:}
%------------------------------------- Figura 1 ----------------------------------------
    \begin{figure} [h!]
        \begin{minipage}[!] {0.4\linewidth}
            \begin{tikzpicture}
                % Marca os vértices do quadrado 
                \coordinate (N) at (0,0);
                \coordinate (O) at (4,0);
                \coordinate (P) at (0,-4);
                \coordinate (Q) at (4,-4);

                % Marca o diametro da circunferência C
                \coordinate (R) at (0,-2);
                \coordinate (S) at (4,-2);
                
                \coordinate[label=right:$L$] (L) at (2,0.5); % Largura do quadrado
                \coordinate[label=right:$C$] (C) at (2.5,-0.5); % Circunferência 
                \coordinate[label=right:$D$] (D) at (2,-2.5); % Diametro da circunferência C

                % Desenha as arestas do quadrado
                \draw[->] (N) -- (O);
                \draw[->] (O) -- (Q);
                \draw[->] (Q) -- (P);
                \draw[->] (P) -- (N);

                % Desenha o diametro (D) da circunferência (C)
                \draw[->] (R) -- (S);

                % Desenha a circunferência (C) com raio 2
                \draw[thick] (2, -2) circle (2);
            \end{tikzpicture}
            \caption{}
        \end{minipage}
        \begin{minipage} [!] {0.4\linewidth}
            % Mostra que pi é a razão entre a circunferência (C) e o diametro (D)
            \[\pi = \frac {C} {D}\]
            
            \[2D < C < 4L = 4D\]
            
            \[2 < \frac {C}{D} < 4 \Rightarrow 2 < \pi < 4\]
        \end{minipage}
    \end{figure}
%------------------------------------- Figura 2 ----------------------------------------
    \begin{figure} [h!]
        \begin{minipage}[!] {0.4\linewidth}
            \begin{tikzpicture}

            % Marca os vértices do quadrado 
            \coordinate (T) at (-2,0);
            \coordinate (U) at (2,0);
            \coordinate (V) at (0,2);
            \coordinate (W) at (0,-2);

            \coordinate[label=right:$L$] (L) at (0,2.5); % Largura do quadrado maior (L)
            \coordinate[label=left:$\frac{D}{2}$] (D) at (0,0.5); % Raio da circunferência (C)
            \coordinate[label=left:$l$] (l) at (1.5,1); % Largura do quadrado menor (l)

            \draw[thick] (-2, -2) rectangle (2, 2); % Quadrado externo
            \draw[thick] (0, 0) circle (2); % Circunferência de raio 2
            \draw[thick] (0, 2) -- (2, 0) -- (0, -2) -- (-2, 0) -- cycle; % Quadrado interno
            \draw[thick] (0, 0) -- (2, 0) -- (0, 2) -- cycle; % Triangulo

            % Desenha as linhas pontilhadas 
            \draw[loosely dotted] (T) -- (U);
            \draw[loosely dotted] (V) -- (W);
            \end{tikzpicture}
            \caption{}
        \end{minipage}
        \begin{minipage} [!] {0.4\linewidth}
            % Faz os cálculos para aproximar pi
            \[4l < C < 4L = 4D\]
            
            \[l^2 = 2(\frac{L}{2})^2 = \cancel{2}(\frac{L^2}{\cancel{4}}) = \frac{D^2}{2} \Rightarrow l =  \frac{D}{\sqrt{2}} = \frac{\sqrt{2}}{2}D\]
    
            {Pelo Teorema de Pitágoras}
    
            \[\therefore \cancel{4} \frac{\sqrt{2}}{\cancel{2}}D < C < 4D\]
            
            \[2\sqrt{2} < \frac{C}{D} < 4 \Rightarrow 2\sqrt{2} < \pi < 4\]
            
            \[\approx 2,82\]
        \end{minipage}
    \end{figure}
%------------------------------------- Figura 3 ----------------------------------------
    \begin{figure} [h!]
        \begin{minipage}[!] {0.4\linewidth}
            \begin{tikzpicture}
            
                % Marca os vertices do triangulo 
                \coordinate (A) at (0, 0);
                \coordinate (B) at (-1, 1.75);
                \coordinate (C) at (1, 1.75);

                \coordinate[label=right:$L$] (L) at (0,2.5); % Largura do quadrado (L)
                \coordinate[label=right:$l$] (l) at (-1.5,0.7); % Lado do hexagono

                \draw[thick] (-2, -2) rectangle (2, 2); % Quadrado
                \draw[thick] (0, 0) circle (2); % Circunferência
                \draw[thick] (0:2) -- (60:2) -- (120:2) -- (180:2) -- (240:2) -- (300:2) -- cycle; % Hexagono
                \draw[thick] (A) -- (B) -- (C) -- cycle; % Triangulo

                % Marca os angulos alfa e beta no triangulo
                \draw[thick] (A) +(0.2,0.1) arc[start angle=0, end angle=100, radius=0.4];
                \node at (0.35, -0.15) {$\alpha^\circ$};
                \draw[thick] (B) +(0.3,-0.5) arc[start angle=-60, end angle=50, radius=0.3];
                \node at (-1, 1.2) {$\beta^\circ$};

                % Desenha as retas pontilhadas
                \draw[loosely dotted] (T) -- (U);
                \draw[loosely dotted] (V) -- (W);
            \end{tikzpicture}
            \caption{}
        \end{minipage}
        \begin{minipage}[!] {0.4\linewidth}
            \[6\alpha = 2\pi \Rightarrow \alpha = \frac{\cancel{2}\pi}{\cancel{6}} = \frac{\pi}{3}\]
    
            {Por outro lado,} 
    
            \[\alpha + 2 \beta = \pi \Rightarrow \cancel{2}\beta = \pi - \alpha = \pi - \frac{\pi}{3} = \cancel{2}\frac{\pi}{3} \Rightarrow \beta = \frac{\pi}{3}\]
    
            {$\therefore \alpha = \beta \Rightarrow$ Triângulo equilátero $\Rightarrow l= \frac{D}{2}$}
        \end{minipage}
    \end{figure} \newpage
%%%%%%%%%%%%%%%%%%%%%%%%%%%%%%%%%%%%% Página 2 %%%%%%%%%%%%%%%%%%%%%%%%%%%%%%%%%%%%%%%%
%------------------------------------- Figura 4 ----------------------------------------
    \begin{figure} [h!]
        \begin{minipage}[!] {0.4\linewidth}
            \begin{tikzpicture}
                \coordinate[label=right:$L$] (L) at (0,2.5); % Largura do hexagono
                \coordinate[label=left:$\frac{D}{2}$] (D) at (0,1); % Raio da circunferência
                \coordinate[label=left:$L$] (l) at (1,0.6); % Largura do hexagono

                % Marca os vertices do triangulo pontilhado
                \coordinate (A) at (0, 0);
                \coordinate (B) at (-1.25, 2.);
                \coordinate (C) at (1.25, 2);
            
                \draw (0:2.35) -- (60:2.35) -- (120:2.35) -- (180:2.35) -- (240:2.35) -- (300:2.35) -- cycle; % Hexagono
                \draw(0,0) circle (2); % Circunferência
                \draw[thick] (-2, -2) rectangle (2, 2); % Quadrado
                \draw[loosely dotted] (A) -- (B) -- (C) -- cycle; % Triangulo

                % Desenha as linhas pontilhadas 
                \draw[loosely dotted] (T) -- (U);
                \draw[loosely dotted] (V) -- (W);
            
            \end{tikzpicture}
            \caption{}
        \end{minipage}
        \begin{minipage}[!] {0.4\linewidth}
            \[L^2 = (\frac{D}{2})^2 + (\frac{L}{2})^2 = \frac{D^2}{4} + \frac{L^2}{4}\]
            
            \[\frac{3}{\cancel{4}}L^2 = L^2 (1-\frac{1}{4}) = \frac{D^2}{\cancel{4}} \Rightarrow L = \frac{D}{\sqrt{3}}\]
        \end{minipage}
    \end{figure}
    
    \[\therefore 6l < C < 6L\] 
    
    \[\cancel{6}\frac{D}{\cancel{2}} < C < 6\frac{D}{\sqrt{3}} = \frac{2.3}{\sqrt{3}}D = \frac{2\sqrt{3}\cancel{\sqrt{3}}}{\cancel{\sqrt{3}}}D\]
    
    \[3 < \frac{C}{D} < 2\sqrt{3} \Rightarrow 3 < \pi < 2\sqrt{3}\]
    
    \[\approx 3,46\]
    \newline

    % Linha do tempo das aproximações de pi
    {250 a.C. Arquimedes polígono 96 $(2^4.6)$ lados limite superior:}
    
    \[\pi \approx \frac{22}{7} = 3,1428571\]
    
    {265 d.C. Liu Hui polígono de 3072 $(2^9.6)$ lados} \newline
    
    {480d.C. Zu Chongzhi polígono 12288 $(2^11.6)$ lados}
    
    \[\pi \approx \frac{355}{113} = 3,1415929\]
    
    \newpage
%%%%%%%%%%%%%%%%%%%%%%%%%%%%%%%%%%%%% Página 3 %%%%%%%%%%%%%%%%%%%%%%%%%%%%%%%%%%%%%%%%
    \section{DUPLICAÇÃO DO QUADRADO}
%------------------------------------- Figura 5 ----------------------------------------
    \begin{figure} [h!]
        \begin{minipage}[!] {0.4\linewidth}
            \begin{tikzpicture}
                \coordinate[label=right:$L$] (L) at (2,4.5); % Largura do quadrado externo
                \coordinate[label=left:$l$] (l) at (3.7,3.5); % Largura do quadrado interno
                
                \def\side{4}
                \draw[thick] (0,0) -- (\side,0) -- (\side,\side) -- (0,\side) -- cycle; % Quadrado externo
                \draw[thick] (\side/2,0) -- (\side,\side/2) -- (\side/2, \side) -- (0, \side/2) -- cycle; % Quadrado interno
    
                % Desenha reta vertical e horizontal, respectivamente
                \draw[loosely dotted] (\side/2,0) -- (\side/2,\side);
                \draw[thick] (0,\side/2) -- (\side,\side/2);
            \end{tikzpicture}
            \caption{}
        \end{minipage}
        \begin{minipage}[!] {0.4\linewidth}
            % Aplicação do Teorema de Pitágoras
            {Pelo Teorema de Pitágoras, tem-se}
    
            \[l^2 = 2 (\frac{L}{2})^2 = \cancel{2} \frac{L^2}{\cancel{4}} \Rightarrow 2l^2 = L^2\] \newline
    
        \end{minipage}
    \end{figure} 
%------------------------------------- Figura 6 ----------------------------------------
    \begin{figure} [h!]
        \begin{minipage}[!] {0.4\linewidth}
            \begin{tikzpicture}

            % Marca os vértices do triângulo retangulo
            \coordinate (A) at (0, 0); 
            \coordinate (B) at (4, 0); 
            \coordinate (C) at (0, 3); 

            % Marca os valores de cada aresta 
            \coordinate[label=left:$1$] (1) at (0,1.5);
            \coordinate[label=left:$1$] (1) at (2,-0.5);
            \coordinate[label=left:$\sqrt{2}$] (H) at (3,2);
            
            \draw[thick] (A) -- (B) -- (C) -- cycle; % Desenha o triangulo
            \end{tikzpicture}
            \caption{}
        \end{minipage}
        \begin{minipage}[!] {0.4\linewidth}
            {Como medir?}
    
            \[mu = L\]
    
            \[nu = l\]
    
            \[\therefore 2 = \frac{L^2}{l^2} = \frac{m^2 \cancel{u^2}}{n^2 \cancel{u^2}} = (\frac{m}{n})^2\] \newline
        \end{minipage}
    \end{figure} 
    
    {Teorema: Não existe nenhum número racional r tal que $r^2 = 2$} \newline
    
    {Demonstração: Por redução ao absurdo, suponha $r= \frac{m}{n}$ tal que:}
    
    \[2=r^2 = \frac{m^2}{n^2} \Rightarrow 2n = \frac{m^2}{n} \Rightarrow \]
    
    \[\Rightarrow 2n - m = \frac{m^2}{n} - m\frac{n}{n} = \frac{m}{n} (m-n) \Rightarrow\] 
    
    \[\Rightarrow \frac{2n-m}{m-n} = \frac{m}{n}\]
%------------------------------------- Figura 7 ----------------------------------------
    \begin{figure} [h!]
        \begin{minipage}[!] {0.4\linewidth}
            \begin{tikzpicture}
                
                \coordinate[label=right:$m$] (m) at (2,4.5); % Largura do quadrado externo
                \coordinate[label=left:$n$] (n) at (3.7,3.5); % Largura do quadrado interno
                \coordinate[label=left:$\frac{m}{2}$] (M) at (3,1.5); 
            
                \def\side{4}
                \draw[thick] (0,0) -- (\side,0) -- (\side,\side) -- (0,\side) -- cycle; % Quadrado externo
                \draw[thick] (\side/2,0) -- (\side,\side/2) -- (\side/2, \side) -- (0, \side/2) -- cycle; % Quadrado interno
                \draw[loosely dotted] (2, 2) -- (4, 2) -- (2, 4) -- cycle; % Triangulo retangulo pontilhado
                
            \end{tikzpicture}
            \caption{}
        \end{minipage}
        \begin{minipage}[!] {0.4\linewidth}
            {Agora, observe que:}
            \[\frac{m}{2} < n \Rightarrow m < 2n\]
    
            \[n < \frac{m}{2} + \frac{m}{2} = m\]
        \end{minipage}
    \end{figure} 
    
    \newpage
    {$\therefore 2n-m > 0$ e $ m-n > 0$ são inteiros \underline{positivos}. Além disso, tem-se:}
    
    \[n < m < 2n \Rightarrow 0 < m-n < n\]
    
    \[n < m < 2n \Rightarrow 2n < 2m = m + m \Rightarrow 2n - m < m\]
    
    
    {$\therefore m, 2n-m,...$ é uma sequência estritamente decrescente de inteiros positivos, \underline{absurdo!}}
%%%%%%%%%%%%%%%%%%%%%%%%%%%%%%%%%%%%% Página 4 %%%%%%%%%%%%%%%%%%%%%%%%%%%%%%%%%%%%%%%%
    \subsection{PROVA ALTERNATIVA:} 

    % Faz uma prova alternativa para a demonstração anterior
    
    {Seja $(\frac{m}{n})^2 = 2$ e suponha, sem perda de generalidade, que m e n são coprimos, ou seja, $(m,n)=1$}
    
    Agora, observe que:
    
    \begin{center}
    $\frac{m^2}{n^2} = 2 \Rightarrow m^2 = 2n^2$ par 
    \newline
    
    $m = 2k + 1 \Rightarrow m^2 = 4k^2 + 4k + 1 = 2k(2k + 2) + 1$ Ímpar, \underline{absurdo!}
    
    {m = 2k par}
    
    $\therefore m^2 = (2k)^2 = \cancel{4}k^2 = \cancel{2}n^2 = 2k^2$ par $\Rightarrow$ n par
    \newline
    
    $\therefore n = 2h \wedge  m = 2k \Rightarrow (m,n) \ge 2$, \underline{absurdo!}
    \newline
    \end{center}
    
    \underline{Colorário:} Ou a duplicação do quadrado é impossível (falso) ou existem magnitudes \underline{incomensuráveis}.
    \newline
    
    Teorema: se N não é um quadrado perfeito, então $\sqrt{N}$ é irracional.
    \newline
    
    Demonstração: Se $\sqrt{N} = \frac{m}{n} \Rightarrow N = \frac{m^2}{n^2} \Rightarrow m^2 = Nn^2 = Nn.n \Rightarrow n | m^2 = m.m \Rightarrow n | m$ se (m, n) = 1 
    \newline
    
    Agora dado que obviamente n | n segue que (m,n) $\ge$ n
    \newline
    
    $\therefore$ \underline{absurdo} ou n = 1 em cujo caso $N = m^2$ quadrado perfeito
    
    \newpage
%%%%%%%%%%%%%%%%%%%%%%%%%%%%%%%%%%%%% Página 5 %%%%%%%%%%%%%%%%%%%%%%%%%%%%%%%%%%%%%%%%
    \section{Área do Círculo}

%------------------------------------- Figura 8 ----------------------------------------
    \begin{figure} [h!]
        \begin{minipage}[!] {0.4\linewidth}
            \begin{tikzpicture}
                \coordinate[label=right:$C$] (C) at (0,2.5); % Circunferência (C)
                \coordinate[label=right:$D$] (D) at (0,-1); Diametro (D) da circunferência (C)
            
                \draw[thick] (0, 0) circle (2); % Desenha a circunferência

                % Desenha a linha vertical e horizontal, repectivamente
                \draw[thick] (T) -- (U);
                \draw[thick] (V) -- (W);
            \end{tikzpicture}
            \caption{}
        \end{minipage}
        \begin{minipage}[!] {0.4\linewidth}
            $\pi = \frac{C}{D}$
        \end{minipage}
    \end{figure}
    
%------------------------------------- Figura 9 ----------------------------------------
    \begin{figure} [h!]
        \begin{minipage}[!] {0.4\linewidth}
            \begin{tikzpicture}
                \def\R{1}

                % Define os pontos para fazer os setores circulares
                \coordinate(N) at (0,0);
                \coordinate(O) at (1,-2);
                \coordinate(P) at (2,0);
                \coordinate(Q) at (3,-2);
                \coordinate(R) at (4,0);
                \coordinate(S) at (5,-2);

                \coordinate[label=right:$\frac{D}{2}$] (D) at (-0.5,-1); % Raio da circunferênia (C)
                \coordinate[label=right:$\frac{C}{2}$] (C) at (2,1.5); % Metade da circunferência (C)

                % Desenha a parte reta dos setores 
                \draw[] (N) -- (O);
                \draw[] (O) -- (P);
                \draw[] (P) -- (Q);
                \draw[] (Q) -- (R);
                \draw[] (R) -- (S);

                % Desenha a parte circular dos setores
                \draw[thick] (2,0) arc[start angle=0, end angle=180, radius=\R];
                \draw[thick] (1,-2) arc[start angle=180, end angle=360, radius=\R];
                \draw[thick] (4,0) arc[start angle=0, end angle=180, radius=\R];
                \draw[thick] (3,-2) arc[start angle=180, end angle=360, radius=\R];
                
            \end{tikzpicture}
            \caption{}
        \end{minipage}
        \begin{minipage}[!] {0.4\linewidth}
            $\approx\frac{C}{2}.\frac{D}{2}$
        \end{minipage}
    \end{figure}
%------------------------------------- Figura 10 ---------------------------------------
    \begin{figure} [h!]
        \begin{minipage}[!] {0.3\linewidth}
            \begin{tikzpicture}
            \draw[thick] (0,0) circle (2); % Circunferência (C) de raio 2

            % Desenha as retas que dividem a circunferência (C) em 8 setores
            \foreach \i in {0,1,...,7} {
                \draw[thick] (0,0) -- ({2*cos(360*\i/8)}, {2*sin(360*\i/8)});
                \draw[thick] (0,0) -- ({2*cos(360*(\i+1)/8)}, {2*sin(360*(\i+1)/8)});
            }
            \end{tikzpicture}
            \caption{}
        \end{minipage}
    \end{figure}
    \newpage
%------------------------------------- Figura 11 ---------------------------------------
    \begin{figure} [h!]
        \begin{minipage}[!] {0.5\linewidth}
            \begin{tikzpicture}
                \def\R{1}

                % Define os pontos para fazer os setores circulares
                \coordinate[label=right:$N$] (N) at (0,0);
                \coordinate[label=right:$O$] (O) at (1,-2);
                \coordinate[label=right:$P$] (P) at (2,0);
                \coordinate[label=right:$Q$] (Q) at (3,-2);
                \coordinate[label=right:$R$] (R) at (4,0);
                \coordinate[label=right:$S$] (S) at (5,-2);
                \coordinate[label=right:$T$] (T) at (6,0);
                \coordinate[label=right:$U$] (U) at (7,-2);
                \coordinate[label=right:$V$] (V) at (8,0);            \coordinate[label=right:$W$] (W) at (9,-2);
                \coordinate[label=right:$\frac{D}{2}$] (D) at (-0.5,-1);
                \coordinate[label=right:$\frac{C}{2}$] (C) at (4,1.5);

                % Desenha a parte reta dos setores 
                \draw[] (N) -- (O);
                \draw[] (O) -- (P);
                \draw[] (P) -- (Q);
                \draw[] (Q) -- (R);
                \draw[] (R) -- (S);
                \draw[] (S) -- (T);
                \draw[] (T) -- (U);
                \draw[] (U) -- (V);
                \draw[] (V) -- (W);

                % Desenha a parte circular dos setores
                \draw[thick] (2,0) arc[start angle=0, end angle=180, radius=\R];
                \draw[thick] (1,-2) arc[start angle=180, end angle=360, radius=\R];
                \draw[thick] (4,0) arc[start angle=0, end angle=180, radius=\R];
                \draw[thick] (3,-2) arc[start angle=180, end angle=360, radius=\R];
                \draw[thick] (6,0) arc[start angle=0, end angle=180, radius=\R];
                \draw[thick] (5,-2) arc[start angle=180, end angle=360, radius=\R];
                \draw[thick] (8,0) arc[start angle=0, end angle=180, radius=\R];
                \draw[thick] (7,-2) arc[start angle=180, end angle=360, radius=\R];
                
            \end{tikzpicture}
            \caption{}
        \end{minipage}
        \begin{minipage}[!] {0.5\linewidth}
            \[\approx \frac{C}{2}.\frac{D}{2}\]
        \end{minipage}
    \end{figure}
    
    Continuando com fatias menores, tem-se:
    
    \[A = \frac{C}{2}.\frac{D}{2} = \frac{C}{D}.\frac{D^2}{4} = \pi.(\frac{D}{2})^2\]
    \newline
    
    Cálculo de $\pi$ usando áreas com hexágonos.
%------------------------------------- Figura 12 ---------------------------------------
    \begin{figure} [h!]
        \begin{minipage}[!] {0.5\linewidth}
            \begin{tikzpicture}
                % Marca os vertices do triangulo 
                \coordinate (A) at (0, 0);
                \coordinate (B) at (-1, 1.75);
                \coordinate (C) at (1, 1.75);

                \coordinate[label=right:$D$] (D) at (0,0); % Diametro da circunferência
                
                \draw[thick] (0, 0) circle (2); % Circunferência
                \draw[thick] (0:2) -- (60:2) -- (120:2) -- (180:2) -- (240:2) -- (300:2) -- cycle; % Hexagono
                \draw[loosely dotted] (A) -- (B) -- (C) -- cycle; % Triangulo pontilhado

                % Marca a extremidades do diametro
                \coordinate (R) at (-2,0);
                \coordinate (S) at (2,0);
                
                \draw[loosely dotted] (R) -- (S); % Desenha o diametro (D)

             \end{tikzpicture}
            \caption{}
        \end{minipage}
        \begin{minipage}[!] {0.5\linewidth}
            \[l = \frac{D}{2} \therefore (\frac{D^2}{2}) = (\frac{l}{2})^2 + h^2 \Rightarrow \]

            \[\Rightarrow h^2 = \frac{D^2}{4} - \frac{D^2}{4^2} = \frac{D^2}{4^2} (4-1) = \frac{3}{4^2}D^2\]

            \[\therefore h= \frac{\sqrt{3}}{4}D\]

            \[\therefore 6.l.h\frac{1}{2} = \cancel{6} \frac{\cancel{D}}{2}.\frac{\sqrt{3}}{\cancel{4}}\cancel{D}.\frac{1}{\cancel{2}} <\pi\frac{\cancel{D^2}}{\cancel{4}}\]

            \[\frac{3\sqrt{3}}{2} < \pi\]
        \end{minipage}
        \newline
    \end{figure}
%%%%%%%%%%%%%%%%%%%%%%%%%%%%%%%%%%%%% Página 6 %%%%%%%%%%%%%%%%%%%%%%%%%%%%%%%%%%%%%%%%
    Por outro lado, tem-se:
%------------------------------------- Figura 13 ---------------------------------------
    \begin{figure} [h!]
        \begin{minipage}[!] {0.5\linewidth}
            \begin{tikzpicture}
                \coordinate[label=right:$L$] (L) at (0,2.5); % Largura do quadrado
                
                % Marca os vertices do triangulo pontilhado
                \coordinate (A) at (0, 0);
                \coordinate (B) at (-1.25, 2.);
                \coordinate (C) at (1.25, 2);
                
                \draw (0:2.35) -- (60:2.35) -- (120:2.35) -- (180:2.35) -- (240:2.35) -- (300:2.35) -- cycle; % Hexagono
                \draw(0,0) circle (2); % Circunferência
                \draw[loosely dotted] (A) -- (B) -- (C) -- cycle; % Triangulo pontilhado

                % Marca a extremidades do diametro
                \coordinate (R) at (-2.5,0);
                \coordinate (S) at (2.5,0);

                \draw[loosely dotted] (R) -- (S); % Desenha o diametro (D)
            \end{tikzpicture}
            \caption{}
        \end{minipage}
        \begin{minipage}[!] {0.5\linewidth}
            \[L= \frac{D}{\sqrt{3}}, \therefore \pi \frac{\cancel{D^2}}{4} < 6.L.h.\frac{1}{2}\]
    
            \[6.\frac{\cancel{D}}{\sqrt{3}}.\frac{\cancel{D}}{\cancel{2}}.\frac{1}{\cancel{2}}\]
    
            \[\therefore \pi < \frac{6}{\sqrt{3}} = \frac{2.3}{\sqrt{3}} = \frac{2.\cancel{\sqrt{3}}.\sqrt{3}}{\cancel{\sqrt{3}}} = 2\sqrt{3}\]
        \end{minipage}
    \end{figure}
    
    Desta maneira resulta:
    
    \[\frac{3\sqrt{3}}{2} \approx 2,6 < \pi < 2\sqrt{3} \approx 3,46\]
    
    \newpage
%%%%%%%%%%%%%%%%%%%%%%%%%%%%%%%%%%%%% Página 7 %%%%%%%%%%%%%%%%%%%%%%%%%%%%%%%%%%%%%%%%
    \section{Papiro de Rhino}
%------------------------------------- Figura 14 ---------------------------------------
    \begin{figure} [h!]
        \begin{minipage}[!] {0.5\linewidth}
            \begin{tikzpicture}
                \draw[thick] (0,0) rectangle (6,6); % Desenha o quadrado externo

                \coordinate[label=right:$D$] (D) at (3,6.5); % Largura do quadrado externo

                % Divide o quadrado em 9 partes iguais
                \foreach \x in {2,4} {
                    \draw[thick] (\x, 0) -- (\x, 6);
                    \draw[thick] (0, \x) -- (6, \x);
                }
            
                % Desenha o círculo inscrito no quadrado externo
                \draw[thick] (3,3) circle (3);

                % Desenha hexagono inscrito no quadrado externo
                \draw[thick,rotate around={30:(3,3)}] (3,6) -- (5.598,4.5) -- (5.598,1.5) -- (3,0) -- (0.402,1.5) -- (0.402,4.5) -- cycle;
            \end{tikzpicture}
            \caption{}
        \end{minipage}
        \begin{minipage}[!] {0.5\linewidth}
            \[A \approx (\frac{D}{3})^2 (9-2) = \frac{7}{9}D^2\]
    
            \[\therefore \pi \frac{\cancel{D^2}}{4} = A \approx \frac{7}{9} \cancel{D^2}\]
    
            \[\therefore \pi \approx \frac{28}{9} = 3,1\]
        \end{minipage}
    \end{figure}
%------------------------------------- Figura 15 ---------------------------------------
    \begin{figure} [h!]
        \begin{minipage}[!] {0.5\linewidth}
            \begin{tikzpicture}
                % Desenha o quadrado externo
                \draw[thick] (0,0) rectangle (6,6);
    
                % Divide o quadrado em 9 partes iguais
                \foreach \x in {2,4} {
                \draw[thick] (\x, 0) -- (\x, 6);  % Linhas verticais
                \draw[thick] (0, \x) -- (6, \x);  % Linhas horizontais
}
                % Desenha uma diagonal cortando o quadrado externo
                \draw[thick] (6,0) -- (0,6);

                % Desenha o semicírculo que encosta nas diagonais do quadrado externo
                \draw[thick] (6,0) arc[start angle=0, end angle=90, radius=6];            \end{tikzpicture}
            \caption{}
        \end{minipage}
        \begin{minipage}[!] {0.5\linewidth}
            \[A \approx (\frac{D}{9})^2(9^2 - \cancel{4}.\frac{9}{\cancel{2}}) = \frac{D^2}{9\cancel{^2}}\cancel{9}(9-2) = \frac{7}{9}D^2\]
        \end{minipage}
    \end{figure}
    
    No entanto, 
    
    \[A \approx (\frac{D}{9}^2 (9^2-\cancel{4}.\frac{9}{\cancel{2}})) = (\frac{D^2}{9} (81-18))\]
    
    \[= (\frac{D}{9})^2.63 \approx (\frac{D}{9})^2.64=(\frac{8}{9}D)^2\]
    
    Quadratura do cículo:
    
    \[L^2 = A = \frac{7}{9}D^2 \Rightarrow L = \sqrt{\frac{7}{9}}D\]
    
    No entanto, alternativamente: 
    
    \[L^2 = A = (\frac{8}{9}D)^2 \Rightarrow L=\frac{8}{9}D\]
    
    Como subproduto obtém-se:
    
    \[\pi \frac{\cancel{D^2}}{4} = A = (\frac{8}{9})^2.\cancel{D^2} \Rightarrow \pi \approx \frac{4.8^2}{9^2} = \frac{2^8}{9^2} = \frac{256}{81} \approx 3,16\]
    
    \newpage
%%%%%%%%%%%%%%%%%%%%%%%%%%%%%%%%%%%%% Página 8 %%%%%%%%%%%%%%%%%%%%%%%%%%%%%%%%%%%%%%%%
\section{PI AT MONTE CARLO}

n= 100000
\newline

m= 1000

var = numeric(m)

for (i in 1:m) var[i] = single\_shot(n)

single\_shot = function(n)
\newline

x= runif(n)

y= runif(n)

z= x.x + y.y

ins= which(z<=1)

pi= 4.length(ins)/n
\newline

plot(x[ins], y[ins], col= "red", pch = 19, cex =0.1, xlim=c(0,1), ylim=c(0,1))

points(x[-ins], y[-ins], cd = "blue")
\newline

mean(var)

sd(var) / sqrt(m)
\newline

Exercício: Determine o "valor médio" de $\pi$. Interprete estatísticamente o resultado.
\newline

Exercício: Implemente Pi via Monte Carlo em C
\newline
%%%%%%%%%%%%%%%%%%%%%%%%%%%%%%%%%%%%% Página 9 %%%%%%%%%%%%%%%%%%%%%%%%%%%%%%%%%%%%%%%%

\#include <stdio.h> 

\#include <stdlib.h> 

\#include <time.h>
\newline

double x,y;
\newline

unsigned long long i, j;
\newline

j= 0;
\newline

stand(time(0));
\newline

for (i=1; i<=n; i++) \{

x = (double) rand() / RAND\_MAX;

y=
\newline

if(x*x + y*y <= 1) \{

j += 1;
\}
\newline
\}
\newline

return 4 * (double) j/n;
\newline

printf("\%1.20f$\backslash$ n", t);

\end{document}