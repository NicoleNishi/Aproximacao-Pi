\documentclass[a4paper, 12pt]{article}

\usepackage[utf8]{inputenc}
\usepackage[brazil]{babel}
% usado para definir a língua que o documento será escrito

\usepackage[lmargin=3cm, tmargin=3cm, rmargin=2cm, bmargin=2cm]{geometry}
% margem

\usepackage[T1]{fontenc}

\usepackage{amsmath, amssymb, inputenc, cancel}
% 

\usepackage{graphicx, xcolor}
% adicionar gráficos, adicionar cores

\usepackage{tikz, tkz-euclide}
% usado para fazer as formas geométricas

\usepackage{scalerel, pict2e, tkz-euclide}
% para ajustar a forma, melhora a qualidade da forma, usado para formar angulos

\begin{document}

%%%%%%%%%%%%%%%%%%%%%%%%%%%%%%%%%%%%% Página 1 %%%%%%%%%%%%%%%%%%%%%%%%%%%%%%%%%%%%%%%%

\section{APROXIMAÇÕES PARA $\pi$}

\textbf{Definição:}

%%%%%%%%%%%%%%%%%%%%%%%%%%%%%%%%%%%%% Figura 1 %%%%%%%%%%%%%%%%%%%%%%%%%%%%%%%%%%%%%%%%
\begin{figure} [h!]
    \begin{minipage}[!] {0.4\linewidth}
        \begin{tikzpicture}
            \coordinate (N) at (0,0);
            \coordinate (O) at (4,0);
            \coordinate (P) at (0,-4);
            \coordinate (Q) at (4,-4);
            \coordinate (R) at (0,-2);
            \coordinate (S) at (4,-2);
            \coordinate[label=right:$L$] (L) at (2,0.5);
            \coordinate[label=right:$C$] (C) at (2.5,-0.5);
            \coordinate[label=right:$D$] (D) at (2,-2.5);
        
            \draw[->] (N) -- (O);
            \draw[->] (O) -- (Q);
            \draw[->] (Q) -- (P);
            \draw[->] (P) -- (N);
            \draw[->] (R) -- (S);
        
            \draw[thick] (2, -2) circle (2);
        \end{tikzpicture}
        \caption{}
    \end{minipage}
    \begin{minipage} [!] {0.4\linewidth}
        \[\pi = \frac {C} {D}\]
        
        \[2D < C < 4L = 4D\]

        \[2 < \frac {C}{D} < 4 \Rightarrow 2 < \pi < 4\]
    \end{minipage}
\end{figure}

%%%%%%%%%%%%%%%%%%%%%%%%%%%%%%%%%%%%% Figura 2 %%%%%%%%%%%%%%%%%%%%%%%%%%%%%%%%%%%%%%%%
\begin{figure} [h!]
    \begin{minipage}[!] {0.4\linewidth}
        \begin{tikzpicture}
        \coordinate (T) at (-2,0);
        \coordinate (U) at (2,0);
        \coordinate (V) at (0,2);
        \coordinate (W) at (0,-2);
        \coordinate[label=right:$L$] (L) at (0,2.5);
        \coordinate[label=left:$\frac{D}{2}$] (D) at (0,0.5);
    
        \draw[thick] (-2, -2) rectangle (2, 2);
        \draw[thick] (0, 0) circle (2);
        \draw[thick] (0, 2) -- (2, 0) -- (0, -2) -- (-2, 0) -- cycle;
        \draw[thick] (0, 0) -- (2, 0) -- (0, 2) -- cycle;
        \draw[loosely dotted] (T) -- (U);
        \draw[loosely dotted] (V) -- (W);
    
        \end{tikzpicture}
        \caption{}
    \end{minipage}
    \begin{minipage} [!] {0.4\linewidth}
        \[4l < C < 4L = 4D\]

        \[l^2 = 2(\frac{L}{2})^2 = \cancel{2}(\frac{L^2}{\cancel{4}}) = \frac{D^2}{2} \Rightarrow l =  \frac{D}{\sqrt{2}} = \frac{\sqrt{2}}{2}D\]

        {Pelo Teorema de Pitágoras}

        \[\therefore \cancel{4} \frac{\sqrt{2}}{\cancel{2}}D < C < 4D\]

        \[2\sqrt{2} < \frac{C}{D} < 4 \Rightarrow 2\sqrt{2} < \pi < 4\]

        \[\approx 2,82\]
    \end{minipage}
\end{figure}

%%%%%%%%%%%%%%%%%%%%%%%%%%%%%%%%%%%%% Figura 3 %%%%%%%%%%%%%%%%%%%%%%%%%%%%%%%%%%%%%%%%
\begin{figure} [h!]
    \begin{minipage}[!] {0.4\linewidth}
        \begin{tikzpicture}
            \coordinate (A) at (0, 0);
            \coordinate (B) at (-1, 1.75);
            \coordinate (C) at (1, 1.75);
            \coordinate[label=right:$L$] (L) at (0,2.5);
            \coordinate[label=right:$l$] (l) at (-1.5,0.7);
            
            \draw[thick] (-2, -2) rectangle (2, 2);
            \draw[thick] (0, 0) circle (2);
            \draw[thick] (0:2) -- (60:2) -- (120:2) -- (180:2) -- (240:2) -- (300:2) -- cycle;
            \draw[thick] (A) -- (B) -- (C) -- cycle;
            
            \draw[thick] (A) +(0.2,0.1) arc[start angle=0, end angle=100, radius=0.4];
            \node at (0.35, -0.15) {$\alpha^\circ$};
            \draw[thick] (B) +(0.3,-0.5) arc[start angle=-60, end angle=50, radius=0.3];
            \node at (-1, 1.2) {$\beta^\circ$};
        
            \draw[loosely dotted] (T) -- (U);
            \draw[loosely dotted] (V) -- (W);
        \end{tikzpicture}
        \caption{}
    \end{minipage}
    \begin{minipage}[!] {0.4\linewidth}
        \[6\alpha = 2\pi \Rightarrow \alpha = \frac{\cancel{2}\pi}{\cancel{6}} = \frac{\pi}{3}\]

        {Por outro lado,} 

        \[\alpha + 2 \beta = \pi \Rightarrow \cancel{2}\beta = \pi - \alpha = \pi - \frac{\pi}{3} = \cancel{2}\frac{\pi}{3} \Rightarrow \beta = \frac{\pi}{3}\]

        {$\therefore \alpha = \beta \Rightarrow$ Triângulo equilátero $\Rightarrow l= \frac{D}{2}$}
    \end{minipage}
\end{figure} \newpage

%%%%%%%%%%%%%%%%%%%%%%%%%%%%%%%%%%%%% Página 2 %%%%%%%%%%%%%%%%%%%%%%%%%%%%%%%%%%%%%%%%
%%%%%%%%%%%%%%%%%%%%%%%%%%%%%%%%%%%%% Figura 4 %%%%%%%%%%%%%%%%%%%%%%%%%%%%%%%%%%%%%%%%
\begin{figure} [h!]
    \begin{minipage}[!] {0.4\linewidth}
        \begin{tikzpicture}
            \coordinate[label=right:$L$] (L) at (0,2.5);
            \coordinate[label=left:$\frac{D}{2}$] (D) at (0,1);
            \coordinate[label=left:$L$] (l) at (1,0.6);
        
            \coordinate (A) at (0, 0);
            \coordinate (B) at (-1.25, 2.);
            \coordinate (C) at (1.25, 2);
        
            \draw (0:2.35) -- (60:2.35) -- (120:2.35) -- (180:2.35) -- (240:2.35) -- (300:2.35) -- cycle;
            \draw(0,0) circle (2);
            \draw[thick] (-2, -2) rectangle (2, 2);
            \draw[loosely dotted] (A) -- (B) -- (C) -- cycle;
        
            \draw[loosely dotted] (T) -- (U);
            \draw[loosely dotted] (V) -- (W);
        
        \end{tikzpicture}
        \caption{}
    \end{minipage}
    \begin{minipage}[!] {0.4\linewidth}
        \[L^2 = (\frac{D}{2})^2 + (\frac{L}{2})^2 = \frac{D^2}{4} + \frac{L^2}{4}\]

        \[\frac{3}{\cancel{4}}L^2 = L^2 (1-\frac{1}{4}) = \frac{D^2}{\cancel{4}} \Rightarrow L = \frac{D}{\sqrt{3}}\]
    \end{minipage}
\end{figure}

\[\therefore 6l < C < 6L\] 

\[\cancel{6}\frac{D}{\cancel{2}} < C < 6\frac{D}{\sqrt{3}} = \frac{2.3}{\sqrt{3}}D = \frac{2\sqrt{3}\cancel{\sqrt{3}}}{\cancel{\sqrt{3}}}D\]

\[3 < \frac{C}{D} < 2\sqrt{3} \Rightarrow 3 < \pi < 2\sqrt{3}\]

\[\approx 3,46\]
\newline

{250 a.C. Arquimedes polígono 96 $(2^4.6)$ lados limite superior:}

\[\pi \approx \frac{22}{7} = 3,1428571\]

{265 d.C. Liu Hui polígono de 3072 $(2^9.6)$ lados} \newline

{480d.C. Zu Chongzhi polígono 12288 $(2^11.6)$ lados}

\[\pi \approx \frac{355}{113} = 3,1415929\]

\newpage

%%%%%%%%%%%%%%%%%%%%%%%%%%%%%%%%%%%%% Página 3 %%%%%%%%%%%%%%%%%%%%%%%%%%%%%%%%%%%%%%%%
\section{DUPLICAÇÃO DO QUADRADO}

%%%%%%%%%%%%%%%%%%%%%%%%%%%%%%%%%%%%% Figura 5 %%%%%%%%%%%%%%%%%%%%%%%%%%%%%%%%%%%%%%%%
\begin{figure} [h!]
    \begin{minipage}[!] {0.4\linewidth}
        \begin{tikzpicture}
            \coordinate[label=right:$L$] (L) at (2,4.5);
            \coordinate[label=left:$l$] (l) at (3.7,3.5);
        
            \def\side{4}
            \draw[thick] (0,0) -- (\side,0) -- (\side,\side) -- (0,\side) -- cycle;
            \draw[thick] (\side/2,0) -- (\side,\side/2) -- (\side/2, \side) -- (0, \side/2) -- cycle;
        
            \draw[loosely dotted] (\side/2,0) -- (\side/2,\side);
            \draw[thick] (0,\side/2) -- (\side,\side/2);
        \end{tikzpicture}
        \caption{}
    \end{minipage}
    \begin{minipage}[!] {0.4\linewidth}
        {Pelo Teorema de Pitágoras, tem-se}

        \[l^2 = 2 (\frac{L}{2})^2 = \cancel{2} \frac{L^2}{\cancel{4}} \Rightarrow 2l^2 = L^2\] \newline

    \end{minipage}
\end{figure} 

{Como medir?}

\[mu = L\]

\[nu = l\]

\[\therefore 2 = \frac{L^2}{l^2} = \frac{m^2 \cancel{u^2}}{n^2 \cancel{u^2}} = (\frac{m}{n})^2\] \newline

{Teorema: Não existe nenhum número racional r tal que $r^2 = 2$} \newline

{Demonstração: Por redução ao absurdo, suponha $r= \frac{m}{n}$ tal que:}

\[2=r^2 = \frac{m^2}{n^2} \Rightarrow 2n = \frac{m^2}{n} \Rightarrow \]

\[\Rightarrow 2n - m = \frac{m^2}{n} - m\frac{n}{n} = \frac{m}{n} (m-n) \Rightarrow\] 

\[\Rightarrow \frac{2n-m}{m-n} = \frac{m}{n}\]

%%%%%%%%%%%%%%%%%%%%%%%%%%%%%%%%%%%%% Figura 6 %%%%%%%%%%%%%%%%%%%%%%%%%%%%%%%%%%%%%%%%
\begin{figure} [h!]
    \begin{minipage}[!] {0.4\linewidth}
        \begin{tikzpicture}
            \coordinate[label=right:$m$] (m) at (2,4.5);
            \coordinate[label=left:$n$] (n) at (3.7,3.5);
            \coordinate[label=left:$\frac{m}{2}$] (M) at (3,1.5);
        
            \def\side{4}
            \draw[thick] (0,0) -- (\side,0) -- (\side,\side) -- (0,\side) -- cycle;
            \draw[thick] (\side/2,0) -- (\side,\side/2) -- (\side/2, \side) -- (0, \side/2) -- cycle;
            \draw[loosely dotted] (2, 2) -- (4, 2) -- (2, 4) -- cycle;
            
        \end{tikzpicture}
        \caption{}
    \end{minipage}
    \begin{minipage}[!] {0.4\linewidth}
        {Agora, observe que:}
        \[\frac{m}{2} < n \Rightarrow m < 2n\]

        \[n < \frac{m}{2} + \frac{m}{2} = m\]
    \end{minipage}
\end{figure} 

\newpage
{$\therefore 2n-m > 0$ e $ m-n > 0$ são inteiros \underline{positivos}. Além disso, tem-se:}

\[n < m < 2n \Rightarrow 0 < m-n < n\]

\[n < m < 2n \Rightarrow 2n < 2m = m + m \Rightarrow 2n - m < m\]


{$\therefore m, 2n-m,...$ é uma sequência estritamente decrescente de inteiros positivos, \underline{absurdo!}}

%%%%%%%%%%%%%%%%%%%%%%%%%%%%%%%%%%%%% Página 4 %%%%%%%%%%%%%%%%%%%%%%%%%%%%%%%%%%%%%%%%
\subsection{PROVA ALTERNATIVA:} 

{Seja $(\frac{m}{n})^2 = 2$ e suponha, sem perda de generalidade, que m e n são coprimos, ou seja, $(m,n)=1$}

Agora, observe que:

\begin{center}
$\frac{m^2}{n^2} = 2 \Rightarrow m^2 = 2n^2$ par 
\newline

$m = 2k + 1 \Rightarrow m^2 = 4k^2 + 4k + 1 = 2k(2k + 2) + 1$ Ímpar, \underline{absurdo!}

{m = 2k par}

$\therefore m^2 = (2k)^2 = \cancel{4}k^2 = \cancel{2}n^2 = 2k^2$ par $\Rightarrow$ n par
\newline

$\therefore n = 2h \wedge  m = 2k \Rightarrow (m,n) \ge 2$, \underline{absurdo!}
\newline
\end{center}

\underline{Colorário:} Ou a duplicação do quadrado é impossível (falso) ou existem magnitudes \underline{incomensuráveis}.
\newline

Teorema: se N não é um quadrado perfeito, então $\sqrt{N}$ é irracional.
\newline

Demonstração: Se $\sqrt{N} = \frac{m}{n} \Rightarrow N = \frac{m^2}{n^2} \Rightarrow m^2 = Nn^2 = Nn.n \Rightarrow n | m^2 = m.m \Rightarrow n | m$ se (m, n) = 1 
\newline

Agora dado que obviamente n | n segue que (m,n) $\ge$ n
\newline

$\therefore$ \underline{absurdo} ou n = 1 em cujo caso $N = m^2$ quadrado perfeito

\newpage

%%%%%%%%%%%%%%%%%%%%%%%%%%%%%%%%%%%%% Página 5 %%%%%%%%%%%%%%%%%%%%%%%%%%%%%%%%%%%%%%%%
\section{Área do Círculo}

%%%%%%%%%%%%%%%%%%%%%%%%%%%%%%%%%%%%% Figura 7 %%%%%%%%%%%%%%%%%%%%%%%%%%%%%%%%%%%%%%%%
\begin{figure} [h!]
    \begin{minipage}[!] {0.4\linewidth}
        \begin{tikzpicture}
            \coordinate[label=right:$C$] (C) at (0,2.5);
            \coordinate[label=right:$D$] (D) at (0,-1);
        
            \draw[thick] (0, 0) circle (2);
            
            \draw[thick] (T) -- (U);
            \draw[thick] (V) -- (W);
        \end{tikzpicture}
        \caption{}
    \end{minipage}
    \begin{minipage}[!] {0.4\linewidth}
        $\pi = \frac{C}{D}$
    \end{minipage}
\end{figure}

%%%%%%%%%%%%%%%%%%%%%%%%%%%%%%%%%%%%% Figura 8 %%%%%%%%%%%%%%%%%%%%%%%%%%%%%%%%%%%%%%%%
\begin{figure} [h!]
    
    \begin{minipage}[!] {0.4\linewidth}
        \begin{tikzpicture}
            \def\R{1}
            
            \coordinate(N) at (0,0);
            \coordinate(O) at (1,-2);
            \coordinate(P) at (2,0);
            \coordinate(Q) at (3,-2);
            \coordinate(R) at (4,0);
            \coordinate(S) at (5,-2);
            \coordinate[label=right:$\frac{D}{2}$] (D) at (-0.5,-1);
            \coordinate[label=right:$\frac{C}{2}$] (C) at (2,1.5);

            \draw[] (N) -- (O);
            \draw[] (O) -- (P);
            \draw[] (P) -- (Q);
            \draw[] (Q) -- (R);
            \draw[] (R) -- (S);

            \draw[thick] (2,0) arc[start angle=0, end angle=180, radius=\R];
            \draw[thick] (1,-2) arc[start angle=180, end angle=360, radius=\R];
            \draw[thick] (4,0) arc[start angle=0, end angle=180, radius=\R];
            \draw[thick] (3,-2) arc[start angle=180, end angle=360, radius=\R];
            
        \end{tikzpicture}
        \caption{}
    \end{minipage}
    \begin{minipage}[!] {0.4\linewidth}
        $\approx\frac{C}{2}.\frac{D}{2}$
    \end{minipage}
\end{figure}
%%%%%%%%%%%%%%%%%%%%%%%%%%%%%%%%%%%%% Figura 9 %%%%%%%%%%%%%%%%%%%%%%%%%%%%%%%%%%%%%%%%
\begin{figure} [h!]
    \begin{minipage}[!] {0.3\linewidth}
        \begin{tikzpicture}
        
        \draw[thick] (0,0) circle (2);
        
        \foreach \i in {0,1,...,7} {
            \draw[thick] (0,0) -- ({2*cos(360*\i/8)}, {2*sin(360*\i/8)});
            \draw[thick] (0,0) -- ({2*cos(360*(\i+1)/8)}, {2*sin(360*(\i+1)/8)});
        }
        \end{tikzpicture}
        \caption{}
    \end{minipage}
\end{figure}
\newpage

%%%%%%%%%%%%%%%%%%%%%%%%%%%%%%%%%%%%% Figura 10 %%%%%%%%%%%%%%%%%%%%%%%%%%%%%%%%%%%%%%%%
\begin{figure} [h!]
    \begin{minipage}[!] {0.5\linewidth}
        \begin{tikzpicture}
            \def\R{1}
            
            \coordinate[label=right:$N$] (N) at (0,0);
            \coordinate[label=right:$O$] (O) at (1,-2);
            \coordinate[label=right:$P$] (P) at (2,0);
            \coordinate[label=right:$Q$] (Q) at (3,-2);
            \coordinate[label=right:$R$] (R) at (4,0);
            \coordinate[label=right:$S$] (S) at (5,-2);
            \coordinate[label=right:$T$] (T) at (6,0);
            \coordinate[label=right:$U$] (U) at (7,-2);
            \coordinate[label=right:$V$] (V) at (8,0);            \coordinate[label=right:$W$] (W) at (9,-2);
            \coordinate[label=right:$\frac{D}{2}$] (D) at (-0.5,-1);
            \coordinate[label=right:$\frac{C}{2}$] (C) at (4,1.5);

            \draw[] (N) -- (O);
            \draw[] (O) -- (P);
            \draw[] (P) -- (Q);
            \draw[] (Q) -- (R);
            \draw[] (R) -- (S);
            \draw[] (S) -- (T);
            \draw[] (T) -- (U);
            \draw[] (U) -- (V);
            \draw[] (V) -- (W);

            \draw[thick] (2,0) arc[start angle=0, end angle=180, radius=\R];
            \draw[thick] (1,-2) arc[start angle=180, end angle=360, radius=\R];
            \draw[thick] (4,0) arc[start angle=0, end angle=180, radius=\R];
            \draw[thick] (3,-2) arc[start angle=180, end angle=360, radius=\R];
            \draw[thick] (6,0) arc[start angle=0, end angle=180, radius=\R];
            \draw[thick] (5,-2) arc[start angle=180, end angle=360, radius=\R];
            \draw[thick] (8,0) arc[start angle=0, end angle=180, radius=\R];
            \draw[thick] (7,-2) arc[start angle=180, end angle=360, radius=\R];
            
        \end{tikzpicture}
        \caption{}
    \end{minipage}
    \begin{minipage}[!] {0.5\linewidth}
        \[\approx \frac{C}{2}.\frac{D}{2}\]
    \end{minipage}
\end{figure}

Continuando com fatias menores, tem-se:

\[A = \frac{C}{2}.\frac{D}{2} = \frac{C}{D}.\frac{D^2}{4} = \pi.(\frac{D}{2})^2\]
\newline

Cálculo de $\pi$ usando áreas com hexágonos.

%%%%%%%%%%%%%%%%%%%%%%%%%%%%%%%%%%%%% Figura 11 %%%%%%%%%%%%%%%%%%%%%%%%%%%%%%%%%%%%%%%%


\[l = \frac{D}{2} \therefore (\frac{D^2}{2}) = (\frac{l}{2})^2 + h^2 \Rightarrow \]

\[\Rightarrow h^2 = \frac{D^2}{4} - \frac{D^2}{4^2} = \frac{D^2}{4^2} (4-1) = \frac{3}{4^2}D^2\]

\[\therefore h= \frac{\sqrt{3}}{4}D\]

\[\therefore 6.l.h\frac{1}{2} = \cancel{6} \frac{\cancel{D}}{2}.\frac{\sqrt{3}}{\cancel{4}}\cancel{D}.\frac{1}{\cancel{2}} <\pi\frac{\cancel{D^2}}{\cancel{4}}\]

\[\frac{3\sqrt{3}}{2} < \pi\]
\newline
%%%%%%%%%%%%%%%%%%%%%%%%%%%%%%%%%%%%% Página 6 %%%%%%%%%%%%%%%%%%%%%%%%%%%%%%%%%%%%%%%%
Por outro lado, tem-se:

\[L= \frac{D}{\sqrt{3}}, \therefore \pi \frac{\cancel{D^2}}{4} < 6.L.h.\frac{1}{2}\]

\[6.\frac{\cancel{D}}{\sqrt{3}}.\frac{\cancel{D}}{\cancel{2}}.\frac{1}{\cancel{2}}\]

\[\therefore \pi < \frac{6}{\sqrt{3}} = \frac{2.3}{\sqrt{3}} = \frac{2.\cancel{\sqrt{3}}.\sqrt{3}}{\cancel{\sqrt{3}}} = 2\sqrt{3}\]

Desta maneira resulta:

\[\frac{3\sqrt{3}}{2} \approx 2,6 < \pi < 2\sqrt{3} \approx 3,46\]

\newpage

%%%%%%%%%%%%%%%%%%%%%%%%%%%%%%%%%%%%% Página 7 %%%%%%%%%%%%%%%%%%%%%%%%%%%%%%%%%%%%%%%%
\section{Papiro de Rhino}

\[A \approx (\frac{D}{3})^2 (9-2) = \frac{7}{9}D^2\]

\[\therefore \pi \frac{\cancel{D^2}}{4} = A \approx \frac{7}{9} \cancel{D^2}\]

\[\therefore \pi \approx \frac{28}{9} = 3,1\]

\[A \approx (\frac{D}{9})^2(9^2 - \cancel{4}.\frac{9}{\cancel{2}}) = \frac{D^2}{9\cancel{^2}}\cancel{9}(9-2) = \frac{7}{9}D^2\]

No entanto, 

\[A \approx (\frac{D}{9}^2 (9^2-\cancel{4}.\frac{9}{\cancel{2}})) = (\frac{D^2}{9} (81-18))\]

\[= (\frac{D}{9})^2.63 \approx (\frac{D}{9})^2.64=(\frac{8}{9}D)^2\]

Quadratura do cículo:

\[L^2 = A = \frac{7}{9}D^2 \Rightarrow L = \sqrt{\frac{7}{9}}D\]

No entanto, alternativamente: 

\[L^2 = A = (\frac{8}{9}D)^2 \Rightarrow L=\frac{8}{9}D\]

Como subproduto obtém-se:

\[\pi \frac{\cancel{D^2}}{4} = A = (\frac{8}{9})^2.\cancel{D^2} \Rightarrow \pi \approx \frac{4.8^2}{9^2} = \frac{2^8}{9^2} = \frac{256}{81} \approx 3,16\]

\newpage

%%%%%%%%%%%%%%%%%%%%%%%%%%%%%%%%%%%%% Página 8 %%%%%%%%%%%%%%%%%%%%%%%%%%%%%%%%%%%%%%%%
\section{PI AT MONTE CARLO}

n= 100000
\newline

m= 1000

var = numeric(m)

for (i in 1:m) var[i] = single\_shot(n)

single\_shot = function(n)
\newline

x= runif(n)

y= runif(n)

z= x.x + y.y

ins= which(z<=1)

pi= 4.length(ins)/n
\newline

plot(x[ins], y[ins], col= "red", pch = 19, cex =0.1, xlim=c(0,1), ylim=c(0,1))

points(x[-ins], y[-ins], cd = "blue")
\newline

mean(var)

sd(var) / sqrt(m)
\newline

Exercício: Determine o "valor médio" de $\pi$. Interprete estatísticamente o resultado.
\newline

Exercício: Implemente Pi via Monte Carlo em C
\newline
%%%%%%%%%%%%%%%%%%%%%%%%%%%%%%%%%%%%% Página 9 %%%%%%%%%%%%%%%%%%%%%%%%%%%%%%%%%%%%%%%%

\#include <stdio.h> 

\#include <stdlib.h> 

\#include <time.h>
\newline

double x,y;
\newline

unsigned long long i, j;
\newline

j= 0;
\newline

stand(time(0));
\newline

for (i=1; i<=n; i++) \{

x = (double) rand() / RAND\_MAX;

y=
\newline

if(x*x + y*y <= 1) \{

j += 1;
\}
\newline
\}
\newline

return 4 * (double) j/n;
\newline

printf("\%1.20f$\backslash$ n", t);

\end{document}